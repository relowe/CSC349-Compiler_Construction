% Don't touch this %%%%%%%%%%%%%%%%%%%%%%%%%%%%%%%%%%%%%%%%%%%
\documentclass[11pt]{article}
\usepackage{fullpage}
\usepackage[left=1in,top=1in,right=1in,bottom=1in,headheight=3ex,headsep=3ex]{geometry}
\usepackage{graphicx}
\usepackage{float}
\usepackage{longtable}

\newcommand{\blankline}{\quad\pagebreak[2]}
%%%%%%%%%%%%%%%%%%%%%%%%%%%%%%%%%%%%%%%%%%%%%%%%%%%%%%%%%%%%%%

% Modify Course title, instructor name, semester here %%%%%%%%


\title{CSC349: Compiler Construction}
\author{Dr. Robert Lowe}
\date{Fall, 2019}

%%%%%%%%%%%%%%%%%%%%%%%%%%%%%%%%%%%%%%%%%%%%%%%%%%%%%%%%%%%%%%

% Don't touch this %%%%%%%%%%%%%%%%%%%%%%%%%%%%%%%%%%%%%%%%%%%
\usepackage[sc]{mathpazo}
\linespread{1.05} % Palatino needs more leading (space between lines)
\usepackage[T1]{fontenc}
\usepackage[mmddyyyy]{datetime}% http://ctan.org/pkg/datetime
\usepackage{advdate}% http://ctan.org/pkg/advdate
\newdateformat{syldate}{\twodigit{\THEMONTH}/\twodigit{\THEDAY}}
\newsavebox{\MONDAY}\savebox{\MONDAY}{Mon}% Mon
\newcommand{\week}[1]{%
%  \cleardate{mydate}% Clear date
% \newdate{mydate}{\the\day}{\the\month}{\the\year}% Store date
  \paragraph*{\kern-2ex\quad #1, \syldate{\today} - \AdvanceDate[4]\syldate{\today}:}% Set heading  \quad #1
%  \setbox1=\hbox{\shortdayofweekname{\getdateday{mydate}}{\getdatemonth{mydate}}{\getdateyear{mydate}}}%
  \ifdim\wd1=\wd\MONDAY
    \AdvanceDate[7]
  \else
    \AdvanceDate[7]
  \fi%
}
\usepackage{setspace}
\usepackage{multicol}
%\usepackage{indentfirst}
\usepackage{fancyhdr,lastpage}
\usepackage{url}
\pagestyle{fancy}
\usepackage{hyperref}
\usepackage{lastpage}
\usepackage{amsmath}
\usepackage{layout}

\lhead{}
\chead{}
%%%%%%%%%%%%%%%%%%%%%%%%%%%%%%%%%%%%%%%%%%%%%%%%%%%%%%%%%%%%%%

% Modify header here %%%%%%%%%%%%%%%%%%%%%%%%%%%%%%%%%%%%%%%%%
\rhead{\footnotesize CSC3490-01 Fall 2019}

%%%%%%%%%%%%%%%%%%%%%%%%%%%%%%%%%%%%%%%%%%%%%%%%%%%%%%%%%%%%%%
% Don't touch this %%%%%%%%%%%%%%%%%%%%%%%%%%%%%%%%%%%%%%%%%%%
\lfoot{}
\cfoot{\small \thepage/\pageref*{LastPage}}
\rfoot{}

\usepackage{array, xcolor}
\usepackage{color,hyperref}
\definecolor{clemsonorange}{HTML}{EA6A20}
\hypersetup{colorlinks,breaklinks,linkcolor=clemsonorange,urlcolor=clemsonorange,anchorcolor=clemsonorange,citecolor=black}

\begin{document}

\maketitle

\blankline

\begin{tabular*}{.93\textwidth}{@{\extracolsep{\fill}}lr}

%%%%%%%%%%%%%%%%%%%%%%%%%%%%%%%%%%%%%%%%%%%%%%%%%%%%%%%%%%%%%%

% Modify information %%%%%%%%%%%%%%%%%%%%%%%%%%%%%%%%%%%%%%%%%
E-mail: \texttt{robert.lowe@maryvillecollege.edu} & Office Phone: 865-981-8169 \\

 Office Hours: MWF 1:00PM -- 2:00PM, TR 10:00AM -- 11:00AM  &  Class Hours: TR 11:00 -- 12:15\\

 Office: SSC 214 & Class Room: SSC 231\\
\hline
\end{tabular*}

\vspace{5 mm}


%%%%%%%%%%%%%%%%%%%%%%%%%%%%%%%%%%%%%%%%%%%%%%%%%%%%%%%%%%%%%%
\section*{Course Description}
This course covers the construction of a compiler for a simple
programming language.  The topics to be studied include formal
grammars, lexical analysis, parsing, and code generation.


%%%%%%%%%%%%%%%%%%%%%%%%%%%%%%%%%%%%%%%%%%%%%%%%%%%%%%%%%%%%%%
\section*{Textbook}
{\em Recursive Descent Compiling} by A.J.T Davie and R. Morrison,
Ellis Horwood, 1981 (PDF on the wiki)


%%%%%%%%%%%%%%%%%%%%%%%%%%%%%%%%%%%%%%%%%%%%%%%%%%%%%%%%%%%%%%
\section*{Goals}
Upon successful completion of this course, students will:
\begin{enumerate}
    \item Understand formal grammars.
    \item Be able to evaluate the quality of a given grammar.
    \item Construct lexical analyzers.
    \item Construct parsers using recursive descent.
    \item Construct compilers.
\end{enumerate}


%%%%%%%%%%%%%%%%%%%%%%%%%%%%%%%%%%%%%%%%%%%%%%%%%%%%%%%%%%%%%%
\section*{Methods of Instruction}
\begin{itemize}
    \item Lecture
    \item Course Projects
    \item Reading
    \item Exams
\end{itemize}


%%%%%%%%%%%%%%%%%%%%%%%%%%%%%%%%%%%%%%%%%%%%%%%%%%%%%%%%%%%%%%
\section*{Grading}
\begin{tabular}{|l|r|}
    \hline
    {\bf Category} & {\bf Weight} \\
    \hline
    Project & 70\%\\
    \hline
    Exams & 20\%\\
    \hline
    Class Participation & 10\% \\
    \hline
\end{tabular}


%%%%%%%%%%%%%%%%%%%%%%%%%%%%%%%%%%%%%%%%%%%%%%%%%%%%%%%%%%%%%%
\section*{Schedule}
\subsubsection*{August 2019}
\begin{tabular}{rrrrrrr}
Su & Mo & Tu & We & Th & Fr & Sa\\
   &    &    &    &  1 &  2 &  3\\
 4 &  5 &  6 &  7 &  8 &  9 & 10\\
11 & 12 & 13 & 14 & 15 & 16 & 17\\
18 & 19 & 20 & 21 & 22 & 23 & 24\\
25 & 26 & 27 & 28 & 29 & 30 & 31\\
\end{tabular}
\begin{description}
\item[Thursday, August 22]
  - Course Introduction

\item[Tuesday, August 27]
  - Introduction \& Mathematical Preliminaries
  \newline - (Chapters 1 \& 2)
\item[Thursday, August 29]
  - Mathematical Preliminaries
  \newline - (Chapter 2)
\end{description}
\hrulefill

\subsubsection*{September 2019}
\begin{tabular}{rrrrrrr}
Su & Mo & Tu & We & Th & Fr & Sa\\
 1 &  2 &  3 &  4 &  5 &  6 & 7\\
 8 &  9 & 10 & 11 & 12 & 13 & 14\\
15 & 16 & 17 & 18 & 19 & 20 & 21\\
22 & 23 & 24 & 25 & 26 & 27 & 28\\
29 & 30 &    &    &    &    &   \\
\end{tabular}
\begin{description}
\item[Tuesday, September 3]
  - Grammatical Preliminaries
  \newline - (Chapter 3)
\item[Thursday, September 5]
  - Grammatical Preliminaries
  \newline - (Chapter 3)

\item[Tuesday, September 10]
  - Testing and Manipulating Grammars
  \newline - (Chapter 4)
\item[Thursday, September 12]
  - Testing and Manipulating Grammars
  \newline - (Chapter 4)
  \newline - {\em Phase 0: Ledgard Programming Due}

\item[Tuesday, September 17]
  - Compiler Construction
  \newline - (Chapter 5)
\item[Thursday, September 19]
  - Compiler Construction
  \newline - (Chapter 5)

\item[Tuesday, September 24]
  - Syntax Analysis
  \newline - (Chapter 6)
\item[Thursday, September 26]
  - Syntax Analysis \& Lexical Analsysis
  \newline - (Chapter 6\&7)
\end{description}

\hrulefill

\subsubsection*{October 2019}
\begin{tabular}{rrrrrrr}
Su & Mo & Tu & We & Th & Fr & Sa\\
   &    &  1 &  2 &  3 &  4 &  5\\
 6 &  7 &  8 &  9 & 10 & 11 & 12\\
13 & 14 & 15 & 16 & 17 & 18 & 19\\ 
20 & 21 & 22 & 23 & 24 & 25 & 26\\ 
27 & 28 & 29 & 30 & 31 &    &   \\
\end{tabular}
\begin{description}
\item[Tuesday, October 1]
    - Lexical Analysis 
    \newline - (Chapter 7)
    \newline - {\em Phase 1: Lexical Analzyer Due}
\item[Thursday, October 3] - Fall Break

\item[Tuesday, October 8] 
    - Review
\item[Thursday, October 10]
    - Midterm Exam

\item[Tuesday, October 15]
    - Syntax Error Diagnosis and Recovery
    \newline - (Chapter 8)
\item[Thursday, October 17]
    - Syntax Error Diagnosis and Recovery
    \newline - (Chapter 8)
    \newline - {\em Phase 2: Parse Due}

\item[Tuesday, October 22]
    - Type Matching
    \newline - (Chapter 9)
\item[Thursday, October 24]
    - Type Matching
    \newline - (Chapter 9)

\item[Tuesday, October 29]
    - Name and Scope Checking
    \newline - (Chapter 10)
\item[Thursday, October 31]
    - Name and Scope Checking
    \newline - (Chapter 10)
    \newline - {\em Phase 3: Generating a Parse Tree Due}

\end{description}

\hrulefill

\subsubsection*{November 2019}
\begin{tabular}{rrrrrrr}
Su & Mo & Tu & We & Th & Fr & Sa\\
   &    &    &    &    &  1 &  2\\
 3 &  4 &  5 &  6 &  7 &  8 &  9\\
10 & 11 & 12 & 13 & 14 & 15 & 16\\
17 & 18 & 19 & 20 & 21 & 22 & 23\\
24 & 25 & 26 & 27 & 28 & 29 & 30\\
\end{tabular}
\begin{description}
\item[Tuesday, November 5]
    - Abstract Machine Design
    \newline - (Chapter 11)
\item[Thurday, November 7]
    - Abstract Machine Design \& Code generation
    \newline - (Chapter 11 \& 12)
    \newline - {\em Phase 4: Symbol Table}

\item[Tuesday, November 12]
    - Code Generation
    \newline - (Chapter 12)
\item[Thurday, November 14] - No Class

\item[Tuesday, November 19]
    - Bootstrapping and Portability
    \newline - (Chapter 13)
\item[Thurday, November 21]
    - Bootstrapping and Portability
    \newline - (Chapter 13)
    \newline - {\em Phase 5 Type Checking Due}

\item[Tuesday, November 26]
    - Project Workshop
\item[Thurday, November 28] - Thanksgiving Break
\end{description}

\hrulefill

\subsubsection*{December 2019}
\begin{tabular}{rrrrrrr}
Su & Mo & Tu & We & Th & Fr & Sa\\
 1 &  2 &  3 &  4 &  5 &  6 &  7\\
 8 &  9 & 10 & 11 & 12 & 13 & 14\\
15 & 16 & 17 & 18 & 19 & 20 & 21\\
22 & 23 & 24 & 25 & 26 & 27 & 28\\
29 & 30 & 31 &    &    &    &   \\
\end{tabular}
\begin{description}
\item[Tuesday, December 3]
    - Project Workshop \& Review
    \newline - {\em Phase 6: Code Generation Due}
\end{description}


%%%%%%%%%%%%%%%%%%%%%%%%%%%%%%%%%%%%%%%%%%%%%%%%%%%%%%%%%%%%%%
\section*{Classroom Policies and Expectations}
In order to ensure a happy and successful semester, I request that you:

\begin{itemize}
\item Silence all cell phones during class. Texting and driving is
deadly. Texting and learning is worse. I know when you are texting,
and it distracts me. Worse, it will distract students around you.
Please be courteous to each other and refrain from surfing the web and
texting during class.
\item Come to class well rested, well fed, and ready to work and learn.
\item Be courteous and respectful of your classmates. Try to make
everyone sitting around you happy to see you. A community of happy
learners is far more successful than a collection of embittered
individuals.
\item Always bring required materials to class.
\item Ask lots of questions! I understand that you like the sound of
my voice. I've always enjoyed it, but you will get much more out of
this course if you ask questions. If you are confused about a topic,
I guarantee you that at least one other person in the room is as well,
so don't be shy about speaking up. Learning is best done as
a dialog, your professors love answering questions!
\end{itemize}

Of course, these guidelines are not quite specific enough for some
questions. More detailed and legalistic policies follow.

\subsection*{Attendance}
You should strive for perfect attendance! Many of the concepts we will
cover can be quite confusing without proper context and guidance. If
you are absent, you will miss out on this, and so you will easily fall
behind. Of course, sometimes life happens and you may need to miss
class. My policy is that you must attend at least 80\% of class
meetings in order to pass the class. If you miss more than 20\% of the
class meetings, unless you have been granted an exception, you will
receive an automatic ``F'' for the course. Also, note the days
of the exams in the schedule. If these dates change, I will notify you
at least week in advance; however, they are not likely to change. If
you must miss an exam period, you will need to make arrangements to
take the exam before the scheduled day. Failure to take an exam on or
before an exam period will result in a grade of zero for that exam.  

\subsection*{Getting Help}
I will do my best to make sure help is readily available to you. You
can come see me during my scheduled office hours. If you see me in my
office and it is not an office hour (a very common occurrence) feel
free to come in and ask any question you like. The same goes for when
you see me around campus. Feel free to approach me and chat with me at
any time! Also, I am available by email and will typically respond
within one business day. I typically do not check my email after
6:00PM or on weekends, however, so there may be a delay during those
times.

\subsection*{Due Dates and Late Work}
Due dates will be provided with each assignment. Except in very rare
emergencies, no individual extensions will be granted. Late work will
receive a grade of zero. This means that getting help is all
the more important! I never grant individual extensions to
assignments, but if lots of people express confusion to me, I am much
more likely to extend the assignment's due date for the entire
class. If no one asks me questions about an assignment, I will assume
everything is going fine and will act accordingly.

\subsection*{Collaboration}
I encourage you to form study groups and consult with each other.
However, each individual in that group must do their own work.
I realize that this can be a tricky dynamic to work with. If you ever
think you may be getting too much help from another student, then you
probably are.

Acceptable collaboration would include:
\begin{itemize}
\item Asking for help on the approach to a problem.
\item Asking for help in locating references in books and/or on the internet.
\item Reading texts together or forming study groups for exams.
\end{itemize}

Unacceptable levels of collaboration include:
\begin{itemize}
\item Showing completed solutions to each other.
\item Asking another student what a specific answer is.
\item Allowing another student to copy an assignment.
\end{itemize}

Any wholesale copying will result in a grade of zero on the assignment for all students involved. This policy is, of course, different for group projects. Unless otherwise instructed, I expect you to submit only your own work for grading.

\subsection*{Academic Honesty}
Academic honesty is a very serious issue. All work that you present as
your own must be your own work. Copying someone else's work in any
way shape or form is unacceptable unless proper attribution is given.
I am, of course, referring to plagiarism. Plagiarism is not always
a blatant act. In fact, most of the time it takes very subtle forms
such as:

\begin{itemize}
\item Presenting an idea that you have read without citing a source.
\item Copying code from a website.
\item Copying steps for a mathematical problem from a website.
\item Using pictures and/or clipart without attribution.
\end{itemize}

As a rule, plagiarism is unspeakably ugly to academic professionals.
Copying without attribution is, in short, a great way to get on your
professor's bad side. Also, the Maryville College policy is that
after 3 instances of plagiarism, you are referred to an academic
honesty board for disciplinary action. So please, do not commit this
heinous deed!

If I catch you in an act of plagiarism, the first offense will result
in a grade of zero and a report will be written and filed with
academic affairs (not to mention a rather unpleasant conversation for
both of us). The second offense will result in a failing grade for the
course along with a second report.

\subsection*{Inclement Weather}
Ah, East Tennessee. Normally pleasant and temperate, but at other
times we are in the midst of a raging whirlpool of blizzards and
tornadoes! Should Maryville College close, an announcement will be
made on the website, in the local media, and via text message (if
you've signed up for the alerts). So long as the college is open,
we will meet and have class. If the college is closed, I will revise
our schedule upon our return.

\subsection*{Students with Disabilities}
Students with a disability requiring accommodations or any student who
believes that they will require accommodations should contact Kim
Ochsenbein in the Academic Support Center located in the lower level
of Thaw Hall (865) 981-8124. Students are encouraged to make contact
before or during the first week of classes.
\end{document}
