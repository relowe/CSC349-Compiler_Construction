\documentclass[]{beamer}
\mode<presentation>
{
  \usetheme{Warsaw}
  \definecolor{mcgarnet}{rgb}{0.38, 0, 0.08}
  \definecolor{mcgray}{rgb}{0.6, 0.6, 0.6}
  \setbeamercolor{structure}{fg=mcgarnet,bg=mcgray}
  %\setbeamercovered{transparent}
}


\usepackage[english]{babel}
\usepackage[latin1]{inputenc}
\usepackage{times}
\usepackage[T1]{fontenc}
\usepackage{tikz}
\usepackage{graphicx}
\usepackage{syntax}
\setlength{\grammarindent}{10em}

\newcommand{\imagesource}[1]{{\centering\hfill\break\hbox{\scriptsize Image Source:\thinspace{\small\itshape #1}}\par}}

\title{06 - Syntax Analysis}


\author{Dr. Robert Lowe\\}

\institute[Maryville College] % (optional, but mostly needed)
{
  Division of Mathematics and Computer Science\\
  Maryville College
}

\date[]{}
\subject{}

\pgfdeclareimage[height=0.5cm]{university-logo}{images/Maryville-College}
\logo{\pgfuseimage{university-logo}}



\AtBeginSection[]
{
  \begin{frame}<beamer>{Outline}
    \tableofcontents[currentsection]
  \end{frame}
}


\begin{document}

\begin{frame}
  \titlepage
\end{frame}

\begin{frame}{Outline}
  \tableofcontents
\end{frame}


% Structuring a talk is a difficult task and the following structure
% may not be suitable. Here are some rules that apply for this
% solution: 

% - Exactly two or three sections (other than the summary).
% - At *most* three subsections per section.
% - Talk about 30s to 2min per frame. So there should be between about
%   15 and 30 frames, all told.

% - A conference audience is likely to know very little of what you
%   are going to talk about. So *simplify*!
% - In a 20min talk, getting the main ideas across is hard
%   enough. Leave out details, even if it means being less precise than
%   you think necessary.
% - If you omit details that are vital to the proof/implementation,
%   just say so once. Everybody will be happy with that.

\section{Syntax Analysis}

\begin{frame}{Syntax Analyzer}
    \begin{itemize}[<+->]
        \item There are two main parts to syntax analysis:
        \begin{itemize}
            \item \textbf{Lexing} - Process the micro-syntax of the
                language. 
            \item \textbf{Syntax Analysis} - Process the context-free
                syntax of the language.
        \end{itemize}
        \item The syntax analyzer can be created directly from the BNF
            specification of a language.
    \end{itemize}
\end{frame}

\begin{frame}{Lexical Analysis Abstraction}
    For now our lexer will consist of the following:
    \begin{itemize}[<+->]
        \item A global variable \texttt{symbol}
        \item \textbf{procedure} \texttt{next\_symbol}
        \begin{enumerate}
            \item Place the next basic symbol in the global variable.
            \item Advance the input stream.
        \end{enumerate}
        \item \textbf{procedure} \texttt{mustbe(s)}
        \begin{enumerate}
            \item if \texttt{s} is the \texttt{symbol}, call
                \texttt{next\_symbol}
            \item Otherwise, report an error.
        \end{enumerate}
        \item \textbf{procedure} \texttt{have(s)}
        \begin{enumerate}
            \item if \texttt{s} is the \texttt{symbol}, call
                \texttt{next\_symbol} and return \texttt{true}.
            \item Otherwise, return \texttt{false}
        \end{enumerate}
    \end{itemize}
\end{frame}

\begin{frame}[fragile]{Coding from BNF}
    \begin{itemize}[<+->]
        \item A production like this: \texttt{< > ::= a<A>}
        \item Would be coded: 
            \newline\texttt{mustbe(``a''); A();}
        \item A production like this: \texttt{< > ::= a<A> | b<B>}
        \item Would be coded:
            \newline \texttt{if have(``a'') then A() else \{
            mustbe(``b''); B() \} }
        \item A production like this: \texttt{< > ::= a<A> | b<B> | c <C> }
        \item Would be coded as:
        \begin{verbatim}
            if( have("a") ) { A(); }
            else if( have("b") ) { B(); }
            else { mustbe("c"); C(); }
        \end{verbatim}
    \end{itemize}
\end{frame}

\begin{frame}[fragile]{Repetition Productions}
    \texttt{< > ::= <A> [b<A>]*} (Where * means repeat ``zero or more times'')

    \pause

    \begin{verbatim}
    do {
        A();
    } while(have("b"));
    \end{verbatim}
\end{frame}

\section{Example: L Programming Language}
\begin{frame}{The Grammar of L}
\begin{grammar}
    <program> ::= <expression>

    <expression> ::= <term> <expression-tail>
    
    <expression-tail> ::= $\lambda$ | '+' <term> <expression-tail>
    
    <term> ::= <factor> <term-tail>
    
    <term-tail> ::= $\lambda$ | '*' <factor> <term-tail>
    
    <factor> ::= <unit> | '(' <expression> ')'
    
    <unit> ::= '0' | '1' | '2' | '3' | '4' | '5' | '6' | '7' | '8'
    | '9'
\end{grammar}

\uncover<2->{\textbf{Activity: }Let's create a syntax analyzer for L!}
\end{frame}

\end{document}


