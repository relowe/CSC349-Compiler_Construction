\documentclass[]{beamer}
\mode<presentation>
{
  \usetheme{Warsaw}
  \definecolor{mcgarnet}{rgb}{0.38, 0, 0.08}
  \definecolor{mcgray}{rgb}{0.6, 0.6, 0.6}
  \setbeamercolor{structure}{fg=mcgarnet,bg=mcgray}
  %\setbeamercovered{transparent}
}


\usepackage[english]{babel}
\usepackage[latin1]{inputenc}
\usepackage{times}
\usepackage[T1]{fontenc}
\usepackage{tikz}
\usepackage{graphicx}

\newcommand{\imagesource}[1]{{\centering\hfill\break\hbox{\scriptsize Image Source:\thinspace{\small\itshape #1}}\par}}

\title{05 - Compiler Construction}


\author{Dr. Robert Lowe\\}

\institute[Maryville College] % (optional, but mostly needed)
{
  Division of Mathematics and Computer Science\\
  Maryville College
}

\date[]{}
\subject{}

\pgfdeclareimage[height=0.5cm]{university-logo}{images/Maryville-College}
\logo{\pgfuseimage{university-logo}}



\AtBeginSection[]
{
  \begin{frame}<beamer>{Outline}
    \tableofcontents[currentsection]
  \end{frame}
}


\begin{document}

\begin{frame}
  \titlepage
\end{frame}

\begin{frame}{Outline}
  \tableofcontents
\end{frame}


% Structuring a talk is a difficult task and the following structure
% may not be suitable. Here are some rules that apply for this
% solution: 

% - Exactly two or three sections (other than the summary).
% - At *most* three subsections per section.
% - Talk about 30s to 2min per frame. So there should be between about
%   15 and 30 frames, all told.

% - A conference audience is likely to know very little of what you
%   are going to talk about. So *simplify*!
% - In a 20min talk, getting the main ideas across is hard
%   enough. Leave out details, even if it means being less precise than
%   you think necessary.
% - If you omit details that are vital to the proof/implementation,
%   just say so once. Everybody will be happy with that.

\section{Compiler Construction}

\begin{frame}{Compiler Phases and Passes}
    \begin{itemize}[<+->]
        \item Three Main Phases:
        \begin{enumerate}
            \item Lexical Analysis
            \item Syntax Analysis
            \item Code Generation
        \end{enumerate}
        \item The phases may require more than one pass.
        \item A recursive descent compiler typically requires only one
            pass.
    \end{itemize} 
\end{frame}

\begin{frame}{Recursive Descent Compiler Design}
    \begin{itemize}[<+->]
        \item Each non-terminal has a corresponding function.
        \item Function calls are mutually recursive.  
        \item That is, functions call each other as they parse the
            program.
        \item For example: \texttt{\textbf{while} <clause> \textbf{do} <clause>}
        \begin{enumerate}
            \item The parser sees the keyword \texttt{while}, and so
                it invokes the \texttt{while()} function.
            \item \texttt{while} then calls the \texttt{clause()} function.
            \item Once \texttt{clause()} returns, \texttt{while}
                checks to see if there is a \texttt{do} keyword.
            \item \texttt{while} then calls the \texttt{clause()}
                function once more.
        \end{enumerate}
    \end{itemize}
\end{frame}

\begin{frame}{Non-Terminal Production Function Design}
    \begin{itemize}[<+->]
        \item The function checks the next lexical symbol. 
        \item Based on the symbol, it then either consumes the symbol
            or it selects a non-terminal production.
        \item Should an unexpected symbol arise, the function should
            report an error.
        \item Let's try designing the functions for the $G$ grammar! (The
        $\mathrm{LL}(1)$ variant):
        \[
        \begin{array}{l}
            S \rightarrow E \\
            E \rightarrow TE'\\
            E' \rightarrow \lambda | + T E' \\
            T \rightarrow FT'\\
            T' \rightarrow \lambda | *FT'\\
            F \rightarrow (E) | U\\
            U \rightarrow 0|1|2|3|4|5|6|7|8|9\\
        \end{array}
        \]
        \item Let's step through some valid and invalid sentences.
    \end{itemize}
\end{frame}

\section{Compiler Layers}
\begin{frame}{Stepwise Refinement}
    \begin{itemize}[<+->]
        \item The compiler process is as follows:
        \begin{enumerate}
            \item Read in the Source
            \item Check the Syntax
            \item Generate the Code
        \end{enumerate}
        \item Each phase of compiler design and construction refines
          these steps, adding more detail as we go.
        \item The easiest approach is to treat view the compiler as
          an ogre (it has layers).
    \end{itemize}
\end{frame}

\begin{frame}{Writing a Recursive Descent Compiler}
    \begin{enumerate}[<+->]
        \item Write a pure syntax analyzer.
        \item Write a lexical analyzer.
        \item Add the context free error diagnosis and recovery.
        \item Add the type checking and type handler.
        \item Add the environment handler and scope checker.
        \item Add the context sensitive error reporting.
        \item Add the data and code address calculation.
        \item Write the code generation.
    \end{enumerate}
\end{frame}

\begin{frame}{Syntax Analysis}
    \begin{itemize}[<+->]
        \item The syntax analyzer is responsible for turning the input
            into a string of basic symbols.
        \item This part of the compiler must be aware of terminals,
            and \textbf{keywords}. 
        \item A keyword is a fixed terminal string, such as
            \texttt{while}, \texttt{if}, etc.
    \end{itemize}
\end{frame}

\begin{frame}{Lexical Analysis}
    \begin{itemize}[<+->]
        \item The lexical analyzer classifies groups of symbols into
            basic constructs.
        \item This is the phase that identifies literals and keywords.
        \item The lexical analyzer reduces the sentence to a series of
            symbols over the $N\cup T$ alphabet.
    \end{itemize}
\end{frame}

\begin{frame}{Context Free Error Diagnosis and Recovery}
    \begin{itemize}[<+->]
        \item This phase basically consists of checking for
            unexpected symbols.
        \item This is a fairly trivial exercise if we have an
            $\mathrm{LL}(1)$ language (or one close to it).
    \end{itemize}
\end{frame}

\begin{frame}{Type Checking}
    \begin{itemize}[<+->]
        \item Type checking validates types used in program
            expressions.
        \item Incompatible types generate errors.
    \end{itemize}
\end{frame}

\begin{frame}{Environment and Scope Checking}
    \begin{itemize}[<+->]
        \item This is symbol table checking.
        \item Verify that all variables are defined in the scope in
            which they are used.
    \end{itemize}
\end{frame}

\begin{frame}{Context Sensitive Error Reporting}
    \begin{itemize}[<+->]
        \item These are errors caused by programs which parse, but are
            meaningless.
        \item Other examples include duplicate names, and other such
            non-syntax related errors.
    \end{itemize}
\end{frame}

\begin{frame}{Machine Abstraction and Code Generation}
    \begin{itemize}[<+->]
        \item An abstract machine is used to compute addresses of
            variables and the like.
        \item This where concepts such as ``stack'' and ``heap'' come
            into play.
        \item Eventually, the abstract machine definition of the code
            is mapped to the real machine during code generation.
        \item These final two layers are the only one with any
            awareness of the underlying computer.  Hence they are
            typically well separated to ensure language portability.
    \end{itemize}
\end{frame}

\begin{frame}{Conclusion}
    \begin{itemize}[<+->]
        \item The process of writing a compiler is about stepwise
            refinement.
        \item The layers are inter-related, however we typically can
            write them through an iterative process.
        \item In the coming weeks, we will study how we make each
            layer work, adding details as we go.
    \end{itemize}
\end{frame}

\end{document}


