\documentclass[]{beamer}
\mode<presentation>
{
  \usetheme{Warsaw}
  \definecolor{mcgarnet}{rgb}{0.38, 0, 0.08}
  \definecolor{mcgray}{rgb}{0.6, 0.6, 0.6}
  \setbeamercolor{structure}{fg=mcgarnet,bg=mcgray}
  %\setbeamercovered{transparent}
}


\usepackage[english]{babel}
\usepackage[latin1]{inputenc}
\usepackage{times}
\usepackage[T1]{fontenc}
\usepackage{tikz}
\usepackage{graphicx}

\newcommand{\imagesource}[1]{{\centering\hfill\break\hbox{\scriptsize Image Source:\thinspace{\small\itshape #1}}\par}}

\title{08 - Symbol Tables}


\author{Dr. Robert Lowe\\}

\institute[Maryville College] % (optional, but mostly needed)
{
  Division of Mathematics and Computer Science\\
  Maryville College
}

\date[]{}
\subject{}

\pgfdeclareimage[height=0.5cm]{university-logo}{images/Maryville-College}
\logo{\pgfuseimage{university-logo}}



\AtBeginSection[]
{
  \begin{frame}<beamer>{Outline}
    \tableofcontents[currentsection]
  \end{frame}
}


\begin{document}

\begin{frame}
  \titlepage
\end{frame}

\begin{frame}{Outline}
  \tableofcontents
\end{frame}


% Structuring a talk is a difficult task and the following structure
% may not be suitable. Here are some rules that apply for this
% solution: 

% - Exactly two or three sections (other than the summary).
% - At *most* three subsections per section.
% - Talk about 30s to 2min per frame. So there should be between about
%   15 and 30 frames, all told.

% - A conference audience is likely to know very little of what you
%   are going to talk about. So *simplify*!
% - In a 20min talk, getting the main ideas across is hard
%   enough. Leave out details, even if it means being less precise than
%   you think necessary.
% - If you omit details that are vital to the proof/implementation,
%   just say so once. Everybody will be happy with that.

\section{Symbol Tables}
\begin{frame}{Purpose}
    \begin{itemize}[<+->]
        \item A symbol table tracks declarations within a program.
        \item If a language has more than one scope, it has more than
            one symbol table.
        \item When a symbol is used in a program, the symbol table(s)
            are checked to ensure that the symbol exists.
        \item Symbol tables account for the unpredictable nature of
            programmer symbols.
    \end{itemize}
\end{frame}

\begin{frame}{Error Detection}
    \begin{itemize}[<+->]
        \item Symbol tables are used to perform error detecting during
            syntax analysis.
        \item \textbf{Re-declaration Errors} -- when a symbol
            which is already in the table is declared again. 
        \item \textbf{Undefined Symbol Errors} -- when a symbol
            is used before it is declared.
        \item \textbf{Type Error} -- when a symbol's type makes it
            invalid in some context.
        \item Note that whether these are errors is dependent upon the
            programming language in question.
    \end{itemize}
\end{frame}

\begin{frame}{Symbol Table Functions}
    \begin{itemize}[<+->]
        \item Symbol tables have two basic operations:
        \begin{description}
            \item[\texttt{declare\_symbol(s, t)}] -- Declare a symbol \texttt{s} of some
                type \texttt{t}.
            \item[\texttt{check\_symbol(s)}] -- Check to see if a symbol exists
                in the table.
        \end{description}
        \item Note that both of these functions should raise an error
            should they detect one.
    \end{itemize}
\end{frame}

\begin{frame}{Language Implications}
    \begin{itemize}[<+->]
        \item When does declaration of symbols occur in a language?
        \item What types exist? (We will do more with type checking
            layer)
        \item How does a language cope with undefined symbols?
        \item When are symbols used in the language?
    \end{itemize}
\end{frame}

\begin{frame}{Integration With Syntax Analyzer}
    \begin{itemize}[<+->]
        \item Any part of the grammar which declares a symbol must add
            the symbol to the appropriate symbol table.
        \item Any part of the grammar which uses a symbol must check
            the symbol table to see if the symbol exists.
        \item Errors in the symbol table should be detected and
            handled as the program is parsed.
    \end{itemize}
\end{frame}

\section{Implementation}
\begin{frame}{Required Information}
    \begin{itemize}[<+->]
        \item The symbol table should maintain some information about
            the symbols in a program.
        \item The name of the symbol.
        \item The type of the symbol.
    \end{itemize}
\end{frame}

\begin{frame}[fragile]{Example Symbol Table}
    \begin{itemize}[<+->]
        \item For example, consider the following C program:
        \begin{verbatim}
int main() 
{
    int var1;
    double var2;
}
        \end{verbatim}
        \item The symbol tables would be:
        \begin{columns}
            \column{0.5\textwidth}
                Global Scope
                \newline\begin{tabular}{|l|l|}
                    \hline
                    \textbf{Symbol} & \textbf{Type} \\
                    \hline
                    \texttt{main} & Integer Function \\
                    \hline
                \end{tabular}
            \column{0.5\textwidth}
                Local Scope for \texttt{main}
                \newline\begin{tabular}{|l|l|}
                    \hline
                    \textbf{Symbol} & \textbf{Type} \\
                    \hline
                    \texttt{var1} & \texttt{int}\\
                    \hline
                    \texttt{var2} & \texttt{double}\\
                    \hline
                \end{tabular}
        \end{columns}
    \end{itemize}
\end{frame}

\begin{frame}{Common Data Structures}
    \begin{itemize}[<+->]
        \item Symbols need to be added and searched efficiently.
        \item Common implementation strategies include:
        \begin{itemize}
            \item Hash Table
            \item Binary Search Trees
            \item Sorted Linked Lists
            \item Sorted Vectors
        \end{itemize}
        \item Hash tables are by far the most common implementation
            method. 
        \item Discuss: Why use hash tables?
    \end{itemize}
\end{frame}

\begin{frame}[fragile]{C++ \texttt{map}}
    \begin{itemize}[<+->]
        \item The STL contains a structure called \texttt{map}, which
            is an associated list of key value pairs.
        \item Maps are declared as follows:
            \newline\verb!map<key_type, val_type> name;!
        \item For a symbol tables \verb!key_type! should probably be
            \texttt{string} and the \verb!val_type! should be some
            appropriate representation of the symbol's type.
        \item There are two critical functions for symbol table use:
            \begin{itemize}
                \item \texttt{operator[]} -- Index operations for
                insertion:
                \newline\verb!table["var1"] = integer_type;!
                \item \texttt{count(key)} -- Count the number of
                    elements matching the key (0 or 1).
            \end{itemize}
    \end{itemize}
\end{frame}

\section{Looplang Symbols}

\begin{frame}[fragile]{Declarations}
    \begin{itemize}[<+->]
        \item When does looplang declare variables?
        \item Looplang declares variables on the first assignment.
        \item What types can looplang symbols take?
        \item Just one, integer.
        \item We could just use a bool to indicate presence of
            a symbol.
        \item Here is pseudocode for when we do a declaration:
        \begin{verbatim}
During the parsing of assignments:
if s does not exist in the table
    table[s] = true
        \end{verbatim}
    \end{itemize}
\end{frame}

\begin{frame}[fragile]{Symbol Use}
    \begin{itemize}[<+->]
        \item When are symbols used in looplang?
        \item They can be in any operand, or on the left hand side of
            assignment.
        \item Operand handling:
        \begin{verbatim}
In operand parsing:
if table.count(s) == 0
   error! 
        \end{verbatim}
    \end{itemize}
\end{frame}


\end{document}


