\documentclass[]{beamer}
\mode<presentation>
{
  \usetheme{Warsaw}
  \definecolor{mcgarnet}{rgb}{0.38, 0, 0.08}
  \definecolor{mcgray}{rgb}{0.6, 0.6, 0.6}
  \setbeamercolor{structure}{fg=mcgarnet,bg=mcgray}
  %\setbeamercovered{transparent}
}


\usepackage[english]{babel}
\usepackage[latin1]{inputenc}
\usepackage{times}
\usepackage[T1]{fontenc}
\usepackage{tikz}
\usepackage{graphicx}

\newcommand{\imagesource}[1]{{\centering\hfill\break\hbox{\scriptsize Image Source:\thinspace{\small\itshape #1}}\par}}

\title{09 - Type Checking}


\author{Dr. Robert Lowe\\}

\institute[Maryville College] % (optional, but mostly needed)
{
  Division of Mathematics and Computer Science\\
  Maryville College
}

\date[]{}
\subject{}

\pgfdeclareimage[height=0.5cm]{university-logo}{images/Maryville-College}
\logo{\pgfuseimage{university-logo}}



\AtBeginSection[]
{
  \begin{frame}<beamer>{Outline}
    \tableofcontents[currentsection]
  \end{frame}
}


\begin{document}

\begin{frame}
  \titlepage
\end{frame}


% Structuring a talk is a difficult task and the following structure
% may not be suitable. Here are some rules that apply for this
% solution: 

% - Exactly two or three sections (other than the summary).
% - At *most* three subsections per section.
% - Talk about 30s to 2min per frame. So there should be between about
%   15 and 30 frames, all told.

% - A conference audience is likely to know very little of what you
%   are going to talk about. So *simplify*!
% - In a 20min talk, getting the main ideas across is hard
%   enough. Leave out details, even if it means being less precise than
%   you think necessary.
% - If you omit details that are vital to the proof/implementation,
%   just say so once. Everybody will be happy with that.

\begin{frame}{Type Productions}
    \begin{itemize}[<+->]
        \item Each expression produces some type.
        \item Types are context sensitive.  Why?
        \item The symbol table is an integral part of type checking.
    \end{itemize}
\end{frame}

\begin{frame}{Type Rules}
    \begin{itemize}[<+->]
        \item Along with the BNF context-free portion of the language,
            we can specify a series of productions for type rules.
        \item For instance, in ledgard, the type rules surrounding
            integer expressions are as follows:  
            \begin{itemize}
                \item $\langle integer-literal \rangle \Rightarrow \{integer\}$
                \item $\{integer\} (+ | - | * | /) \{integer\} \Rightarrow \{integer\}$ 
            \end{itemize}
        \item This forms a grammar which an be readily checked
            for correctness.
        \item Type checking is typically performed either during
            syntax analysis or on a parse tree.
        \item Each production in the language has a type production.
            In ledgard, the possible results of a production are:
            \begin{itemize}
                \item \textbf{Simple Type} - integer, boolean
                \item \textbf{Type} - integer, boolean, array
                \item \textbf{void} - Nothing is returned
            \end{itemize}
    \end{itemize}
\end{frame}

\begin{frame}{Type Representation}
    \begin{itemize}[<+->]
        \item Types can be represented in much the same way as
            lexemes.
        \item As the parse tree is constructed, the types are
            checked, and the types productions are noted in the tree.
    \end{itemize}
\end{frame}

\begin{frame}{Type Error Reporting and Recovery}
    \begin{itemize}[<+->]
        \item Error checking and reporting is analogous to that of
            parsing.
        \item Recovery is typically easier, because for each
            production we typically know a valid type production.
        \item Reporting can be tricky in situations with multiple
            legal productions.
        \item The basic procedure is this:
        \begin{enumerate}
            \item Check the types of the current parse tree node.
            \item If there is an error, report it.
            \item Mark the current node as if the production
                succeeded and continue (or maybe just stop all
                together!).
        \end{enumerate}
    \end{itemize}
\end{frame}

\begin{frame}{Exercise: Looplang Type Checking}
    \begin{itemize}[<+->]
        \item How many types does looplang have?
        \item Let's write the type rules for looplang!
        \item Now, let's add type checking to looplang.
    \end{itemize}
\end{frame}

\end{document}


