\documentclass{beamer}
\mode<presentation>
{
  \usetheme{Warsaw}
  \definecolor{mcgarnet}{rgb}{0.38, 0, 0.08}
  \definecolor{mcgray}{rgb}{0.6, 0.6, 0.6}
  \setbeamercolor{structure}{fg=mcgarnet,bg=mcgray}
  %\setbeamercovered{transparent}
}


\usepackage[english]{babel}
\usepackage[latin1]{inputenc}
\usepackage{times}
\usepackage[T1]{fontenc}
\usepackage{tikz}
\usepackage{graphicx}

\newcommand{\imagesource}[1]{{\centering\hfill\break\hbox{\scriptsize Image Source:\thinspace{\small\itshape #1}}\par}}

\title{01 - Introduction and Math Preliminaries}


\author{Dr. Robert Lowe\\}

\institute[Maryville College] % (optional, but mostly needed)
{
  Division of Mathematics and Computer Science\\
  Maryville College
}

\date[]{}
\subject{}

\pgfdeclareimage[height=0.5cm]{university-logo}{images/Maryville-College}
\logo{\pgfuseimage{university-logo}}



\AtBeginSection[]
{
  \begin{frame}<beamer>{Outline}
    \tableofcontents[currentsection]
  \end{frame}
}


\begin{document}

\begin{frame}
  \titlepage
\end{frame}

\begin{frame}{Outline}
  \tableofcontents
\end{frame}


% Structuring a talk is a difficult task and the following structure
% may not be suitable. Here are some rules that apply for this
% solution: 

% - Exactly two or three sections (other than the summary).
% - At *most* three subsections per section.
% - Talk about 30s to 2min per frame. So there should be between about
%   15 and 30 frames, all told.

% - A conference audience is likely to know very little of what you
%   are going to talk about. So *simplify*!
% - In a 20min talk, getting the main ideas across is hard
%   enough. Leave out details, even if it means being less precise than
%   you think necessary.
% - If you omit details that are vital to the proof/implementation,
%   just say so once. Everybody will be happy with that.

\section{Introduction to Compilers}

\begin{frame}{What is a compiler?}
    
    A compiler ...
    \begin{itemize}[<+(1)->]
       \item verifies the validity of the source program.
       \item translate a source program into an object program.
       \item translates a source program without changing its semantic meaning.
    \end{itemize}
\end{frame}


\begin{frame}{Program Stages}
    \begin{itemize}[<+(1)->]
        \item Compile Time
        \begin{itemize}
            \item Lexicographical Properties of the Program
            \item Validation
            \item Code Production
        \end{itemize}

        \item Load Time
        \begin{itemize}
            \item Loading and linking of shared libraries
            \item Relocation of Code
        \end{itemize}

        \item Run Time
        \begin{itemize}
            \item Program Execution
            \item Dynamic Behavior of the Program
        \end{itemize}
    \end{itemize}
\end{frame}


\begin{frame}{Phases of Compilation}
    \begin{enumerate}[<+->]
        \item Lexical Analysis
        \item Syntax Analysis
        \item Code Generation
    \end{enumerate}
\end{frame}


\begin{frame}{Lexical Analysis}
    \begin{itemize}[<+->]
        \item Analysis of {\bf microsyntax} of a language.
        \item Breaking a program into tokens (aka basic symbols or lexemes)
        \item Not always trivial! (Context can alter tokens)
        \newline Consider the following Fortran:
        \begin{itemize}
            \item {\tt DO 1 I=1,12}
            \item {\tt DO 1 I=1.12}
        \end{itemize}
    \end{itemize}
\end{frame}

\begin{frame}{Syntax Analysis}
    \begin{itemize}[<+->]
        \item Analyzes the structure of the program.
        \item Validates structure.  (i.e. Do \{ \} match?)
        \item Results in a {\bf parse tree} representation of the program.
    \end{itemize}
\end{frame}

\begin{frame}{Code Generation}

    The code generator ...
    \begin{itemize}[<+->]
        \item traverses the parse tree.
        \item generates object code as it descends the tree.
        \item optimizes object code.
    \end{itemize}
\end{frame}

\begin{frame}{Recursive Descent Compiling}
    \begin{itemize}[<+->]
        \item Each major language structure has a corresponding recognizer routine.
        \item These methods call each other as needed.
        \item As the methods get called, they construct a parse tree.
        \item Errors are detected as the recognizers execute.
        \item Limited in scope to $\mathrm{LL}(1)$ languages.
    \end{itemize}
\end{frame}

\section{S-Algol}

\begin{frame}{Language Properties}
    \begin{itemize}[<+->]
        \item ALGOL Inspired Language
        \item Sequence Level Scoping
        \item Types are Inferred at Declaration
        \item Vectors for Lists of Variables
        \item Structures
        \item Procedures
        \item Designed as a Teaching Language
        \item Powerful Enough for Systems Programming
    \end{itemize}
\end{frame}

\begin{frame}[fragile]{Variable Declarations}

    \begin{verbatim}
let x := 1              !has type int i.e.integer
let y := 2.7            !has type real
let switch := x<pi      !has type bool i.e. boolean
let name := "Bill"      !has type string
let e=2.71828           !real constant
let lbl := "here"       !has type cstring
    \end{verbatim}
\end{frame}

\begin{frame}[fragile]{Structures}
\begin{verbatim}
structre identifier(cstring name ;real val)
let var := identifier("x", 2.14)
\end{verbatim}
\end{frame}

\begin{frame}[fragile]{Procedures}
\begin{verbatim}
procedure count(cint s,e)
begin
  let x := s
  while x <= e do
  begin
    write x
    x := x + 1
  end
end

procedure convert(cint L,S,D->real)
    L+S/20+D/240  
\end{verbatim}
\end{frame}


\section{Math Preliminaries}
\begin{frame}{Closure}
\begin{itemize}[<+->]
    \item {\bf Closure} - A sequence of objects which close a set of
    objects.
    \item We often speak of closure of grammars, languages, and sets.
    \item Computing closures reveal vital information about
    a language.
    \item  We will get more formal with closures later.
\end{itemize}
\end{frame}

\begin{frame}{Alphabets and Languages}
\begin{itemize}[<+->]
    \item An alphabet, $A$, is a finite set of symbols.
    For example:
    \[
        A=\{a, b, c\}
    \]
    \item A language, $L$, is the set of sequences or strings over
    some alphabet. For example, we may have:
    \[
        L=A \times A
    \]
    \item Expanding the above language yields:
    \[
        L=\{aa, ab, ac, ba, bb, bc, ca, cb, cc\}
    \]
\end{itemize}
\end{frame}

\begin{frame}{Strings of Length $k$}
\begin{itemize}[<+->]
    \item Often we wish to express arbitrarily long strings of
        alphabet $A$.
    \item Strings of length $k$ are written as $A^k$ where $k$
        represents the number of times the Cartesian product is applied to
        $A$ and itself.
    \item For example:
    \[
    A^3 = A \times A \times A
    \]
    Represents all strings consisting of $3$ symbols from $A$.
    \item $A^0$ is the empty string, we often give it the special
    symbol $\lambda$
\end{itemize}
\end{frame}

\begin{frame}{Reflexive Transitive Closure}
\begin{itemize}[<+->]
    \item Suppose we take $A^k$ for all values $k=[0\ldots \infty]$
    \item The union of every such set of sequences is called $A^*$
    \[
        A^* = \bigcup\limits_{n=0}^{\infty} A^n
    \]
    \item $A^*$ is the reflexive transitive closure of $A$
    \item Also known as the Kleene closure (after its definer, Stephen
    Kleene).
    \item $A^*$ is the language consisting of every possible string
    over the alphabet $A$.
\end{itemize}
\end{frame}

\begin{frame}{Transitive Closure}
\begin{itemize}[<+->]
    \item Sometimes, we wish to exclude $\lambda$ from the set of
    strings.  This is the $A^+$ closure.
    \item $A^+ = A^* - \lambda$
    \item Fully stated:
    \[
        A^+ = \bigcup\limits_{n=1}^{\infty} A^n     
    \]
    \item This is called the Transitive Closure of $A$
\end{itemize}
\end{frame}

\begin{frame}{Concatenation}
\begin{itemize}[<+->]
    \item {\bf Concatenation} is an operator which combines two strings.
    \item Concatenation (represented as a $.$) defines an algebra over
        $A^*$.
    \item $\lambda$ is the unit element since $\forall s \in A^*$:
        \[
            s . \lambda = \lambda . s = s
        \]
    \item Concatenation is associative, but not commutative.
    \item {\bf Closure} is the smallest set containing a given basic
        set closed under certain operations.
    \item Hence $A^*$ is the reflexive transitive closure of $A$ under
        the operation of concatenation.
    \item Also, $A^+$ is the transitive closure of $A$ under the 
        operation of concatenation.
\end{itemize}
\end{frame}

\begin{frame}{Languages and Strings}
\begin{itemize}[<+->]
    \item If we think of a programming language, $L$ is the set of all
        strings which form a valid program.
    \item Usually, we think of the alphabet of a programming
        language as being its individual lexemes.
    \item For almost all programming languages:
        \[
            L \subset A^*
        \]
        Where $A$ is the alphabet of lexemes of the language.
    \item A compiler, is therefore formally stated as a device which,
    given a string $s$, a source language $L$, and an object language
    $L'$:
    \newline $\forall s\in A^*$:
    \begin{enumerate}
        \item Decides $s \stackrel{?}{\in} L$
        \item Computes the function $L \mapsto L'$ on $s$
    \end{enumerate}
\end{itemize}
\end{frame}

\end{document}


